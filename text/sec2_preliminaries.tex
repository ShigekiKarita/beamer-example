\section{Preliminaries}
\begin{frame}\frametitle{Compressed Sensing}
Candes \& Donoho らが提案する信号や画像を低次元に再構築する手法
\begin{equation}
    f = \Phi x (\Phi: N \times N, x: N \times 1)
\end{equation}
$x$ 中で $K$ 個の係数しか大きな値を取らないとき,$x$ を $f$ の $K$ 疎表現と呼ぶ.
\begin{eqnarray}
    y & = & \Omega f = \Theta x (\Omega: M \times N, y: M \times 1) \\
    \Theta & = & \Omega \Phi
\end{eqnarray}
上記の低次元空間への射影 $\Theta$ を sensing matrix と呼ぶ.
$\Theta$ は Ristricted Isometry Property
\begin{eqnarray}
    (1 - \delta_k) ||c||^2_2 & \leq & || \Theta_c ||^2_2 \leq (1 + \delta_k) ||c||^2_2 \\
    0 & < & \delta_k < 1
\end{eqnarray}
を満たすとする,ただし $c$ はランダムな $K$ 疎ベクトルである.
最適な疎表現 $x$ はL0最適化だが,NP完全問題なのでL1ノルム最適化
\begin{equation}
    \arg \min ||x||_1 s.t. y = \Theta x
\end{equation}
として近似して解く。
L1ノルム最適化アルゴリズムとしては以下が知られる
\begin{itemize}
    \item 貪欲法 :
        Match Pursuit (MP),
        Orthogonal MP (OMP)
    \item 凸緩和最適化 :
        Gradient Projection for Sparse Reconstruction (GPSR),
        Total Variation (TV),
        Smoothed-L0 (SL0)
    \item 組み合わせ最適化 : Iwen の手法
\end{itemize}
\end{frame}


\begin{frame}\frametitle{K-SVD}
訓練データセット $\{ x \}$ から疎な辞書モデルを構築できるアルゴリズム.
最適化問題としての定式化は以下
\begin{eqnarray}
    \min \{||x||_0\} &s.t.& \min ||Y-DX||^2_F \leq \varepsilon (i = 1, \dots, N) \\
    ||Y-DX||^2_F &=& ||Y - \sum D_k x_k||^2_F \\
    &=& ||(Y - \sum_{k \neq j} d_k x_k) - d_j x_j||^2_F \\
    &=& ||E_j - d_j x_j||^2_F
\end{eqnarray}
ただし $F$ はフロベニスノルム $||A||_F = \sqrt{\sum{A^2_{i,j}}}$
\end{frame}
